\section{Gestione dei dati}

Il sito presenta 3 tipi di contenuti che possono essere modificati da parte dagli utenti autorizzati:
\begin{itemize}
\item \textbf{Immagini}: le foto che vengono visualizzate nella galleria fotografica;
\item \textbf{News}: le notizie che vengono mostrate sia nell'homepage sia nell'apposita sezione;
\item \textbf{Utenti}: gli amministratori del sito possono gestire gli utenti autorizzati a modificare i contenuti dinamici del sito.
\end{itemize}
Questi dati vengono memorizzati negli omonimi file \texttt{.xml} presenti nella cartella \texttt{data}.

\subsection{XMLSchema}
Per verificare la validità dei dati sono stati creati degli appositi XMLSchema che definisco i vari tag che possono comparire nei file \texttt{.xml} e, quando necessario, i vincoli di unicità. \\
Gli schemi \texttt{utenti.xsd} e \texttt{news.xsd} definiscono infatti, dei vincoli di unicità per gli id degli elementi e, nel caso degli utenti, viene definito anche il vincolo di unicità dell'indirizzo email.

\subsection{XSLT}
Per visualizzare il contenuto erano stati crearti dei template XSLT, tuttavia è risultato più pratico fare la conversione da XML a HTML sfruttando il gli script Perl e di conseguenza non sono stati usati.


