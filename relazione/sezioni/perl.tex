\section{Perl}

Perl viene usato per gestire la visualizzazione e la gestione delle informazioni dinamiche.
Gli script CGI si possono divire in due sezioni:
\begin{itemize}
\item Quelli che generano le pagine visualizzabili dai visitatori;
\item Quelli che generano le pagine legate all'amministrazione del sito;
\end{itemize}


\subsection{Pagine per i visitatori}

Queste pagine vengono generate leggendo i dati dai relativi file XML e vengono create con uno stile analogo a quello delle altre pagine del sito.
L'utente in questo caso può solamente visualizzare i contenuti, senza poterli modificare.\\
Le pagine che rientrano in questa categoria sono:
\begin{itemize}
\item \texttt{index.cgi}: Rappresenta la pagina iniziale del sito, vengono visualizzate la ultime 3 news inserite e per questo è stato necessario utilizzare il Perl per generarla;
\item \texttt{news.cgi}: Rappresenta l'archivio delle news, contiene tutte le news presenti nel file \texttt{news.xml}, ordinate in ordine cronologico inverso;
\item \texttt{dettaglioNews.cgi}: Viene utilizzata per visualizzare una singola notizia;
\item \texttt{foto.cgi}: Galleria di fotografie realizzata a partire dai dati contenuti nel file \texttt{immagini.xml}.
\end{itemize}


\subsection{Pagine di amministrazione}

Per accedere a queste pagine è necessario effettuare il login in modo da essere sicuri che solo le persone autorizzate riescano ad accedere a questa area del sito.
Una volta che l'utente ha effettuato il login potrà gestire le news e le foto.
Inoltre, se l'utente autenticato è un amministratore del sito potrà anche gestire gli utenti.\\
Per offrire queste funzionalità sono stati creati vari script Perl che permettono di inserire, cancellare e, quando ha senso, modificare i dati presenti nei vari file \texttt{.xml}.\\
A questa tipologia di pagine appartengo anche le pagine generate per gestire il login/logout e la pagina generata dallo script \texttt{noauth.cgi}, che viene mostrata quando l'utente tenta di accedere ad una pagina senza averne il permesso.

\subsection{Funzioni comuni a più pagine}

Per semplificare la gestione del codice ed aumentarne la manutenibilità sono state create 3 "librerie":
\begin{itemize}
\item \texttt{customHtmlFunction.pl}: che contiene le funzioni relative alla stampa delle parti HTML comuni a più pagine, come per esempio la navbar;
\item \texttt{sessionHelper.pl}: che contiene le funzioni relative alla gestione delle sessioni;
\item \texttt{funzioni.pl}: che contiene le funzioni \textit{general purpose} per il controllo dell'input e gestione dei file.
\end{itemize}


\subsection{Gestione della sessione}

Se il login avviene con successo viene creata una sessione sfruttando le funzioni messe a disposizione dalla libreria \texttt{sessionHelper.pl}.\\
Tra le varie funzioni offerte è presente la funzione \texttt{createSession()} che riceve come parametri:
\begin{itemize}
\item L'indirizzo email dell'utente che ha effettuato il login;
\item Il livello di autorizzazione dell'utente;
\item Un riferimento ad un oggetto \texttt{CGI}, necessario per creare il cookie.
\end{itemize}
Essendo l'id della sessione comunicato al cliente tramite cookie è necessario che il browser del client abbia la recezione dei cookie abilitati, altrimenti non sarà possibile fare il login.\\
Il livello di autorizzazione memorizzato nella sessione può assumere i seguenti valori:
\begin{itemize}
\item \textbf{1}: l'utente corrente è un utente dello staff e di conseguenza può gestire le news e le foto;
\item \textbf{2}: l'utente corrente è un amministratore del sito che può gestire anche gli utenti.
\end{itemize}
Il livello di autorizzazione viene utilizzato nei vari script per verificare che l'utente abbia i permessi necessari per visualizzare la pagina. Nel caso l'utente non disponga dei permessi necessari verrà segnalato l'errore indirizzandolo alla pagina \texttt{noauth.cgi}.\\
La sessione viene distrutta o dopo un periodo di inattività oppure quando l'utente sceglie di effettuare il logout.
